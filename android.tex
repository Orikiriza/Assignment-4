\documentclass[12pt]{article}
\usepackage[top=2.5cm, bottom=2.5cm, left=3cm, right=3cm]{geometry}
\usepackage{graphicx}
\begin{document}

\begin{Huge}
\begin{center}
\begin{normalsize}
\textbf{MAKERERE UNIVERSITY }\\


\textbf{COLLAGE OF COMPUTING AND INFORMATION SCIENCES} \\
\textbf{SCHOOL OF COMPUTING AND INFORMATICS TECHNOLOGY} \\
\textbf{DEPARTMENT OF COMPUTER SCIENCE} \\
\textbf{BACHELOR OF SCIENCE IN COMPUTER SCIENCE} \\
\textbf{BIT 2207 RESEARCH METHODOLOGY} \\
\textbf{Course Work: Assignment 4}\\
\end{normalsize}
\end{center}
\end{Huge}

\begin{center}
\begin{tabular}{|l|l|l|c|}
\hline NAME  & REG NO & STD NO \\\hline
OSCAR ORIKIRIZA & 16/U/1055 & 216000566 \\\hline
\end{tabular}\paragraph{•}
Lecturer: ERNEST MWEBAZE \\
\end{center}

\newpage

\begin{center}
\textbf{Android Application Development using Android Studio}\\

\paragraph{•}


\end{center}



\section{Introduction}
\paragraph{•}
This era is very great and exiting for mobile developers. Android [1] is an open source architecture that includes the Operating system, application framework, Linux kernel, middleware and application along with a set of API libraries for writing mobile applications that can give look, feel, and function of mobile handsets.

\paragraph{Background}
\paragraph{•}
Android mobile operating system has begun its version history with the release of the Android beta version in November 2007. Android 1.0 (First version),the first commercial version was released in September 2008. As we all know that Android is introduce byGoogleand theOpen Handset Alliance(OHA), and since its initial release, we have seen a number of updates to its base operating system.

\subparagraph{Android studio}
\paragraph{•}
Android Studio[2] is the official integrated development environment (IDE) for Android platform development.

\section{Overview}
\paragraph{Android Architecture[3]}
We studied the Android operating system architecture.
Android system is a Linux-based system, Android operating
system is a stack of software components which is roughly
divided into five sections and four main layers as shown
below in the architecture diagram.Each layer of the lower
encapsulation, while providing call interface to the upper.

\section{References}
\paragraph{•}
[1] http://www.tutorialspoint.com/android/android_architecture.html
\paragraph{•}
[2] http://developer.android.com/guide/basics/what-isandroid.html
\paragraph{•}
[3] http://www.slideshare.net/VijayRastogi/ppt2-intro-androidarchitecturecomponentsd6

\end{document}